\documentclass[12pt]{article}
\usepackage[utf8]{inputenc}
\usepackage{geometry}
\geometry{margin=2cm}
\usepackage{graphicx}
\usepackage{hyperref}
\usepackage{media9}
\usepackage{listings}
\usepackage{xcolor}
\usepackage{tcolorbox}
\usepackage{url}

% Configuración de código
\lstset{
    basicstyle=\ttfamily\small,
    keywordstyle=\color{blue}\bfseries,
    commentstyle=\color{green!60!black},
    stringstyle=\color{red},
    numberstyle=\tiny\color{gray},
    numbers=left,
    stepnumber=1,
    breaklines=true,
    frame=single,
    backgroundcolor=\color{gray!5},
    tabsize=4,
    captionpos=b
}

\begin{document}

% ========================== PORTADA ==========================
\begin{titlepage}
    \centering
    {\Large \textbf{Programación de Servicios y Procesos}\par}
    \vspace{2cm}
    {\Huge \textbf{Emulador de Mario Kart}\par}
    \vspace{0.5cm}
    {\LARGE Carrera de coches con Interfaz Gráfica en JavaFX\par}
    \vspace{2cm}
    \noindent\rule{12cm}{0.8pt}
    \vspace{2cm}
    {\Large \textbf{Autor:}\par}
    \vspace{0.3cm}
    {\large Santi Martínez\par}
    \vspace{10cm}
    {\large \today}
\end{titlepage}

\newpage
\renewcommand{\contentsname}{ÍNDICE}
\tableofcontents

% ========================== INTRODUCCIÓN ==========================
\newpage
\section{Introducción}
    Este proyecto consiste en un emulador simplificado de una carrera de \textbf{Mario Kart} desarrollado en Java utilizando \textbf{JavaFX} para la interfaz gráfica y los conceptos fundamentales de \textbf{programación concurrente} con hilos (\texttt{Thread}).

    \vspace{0.5cm}
    El objetivo principal es simular una carrera entre cinco icónicos personajes del universo Mario Kart (Mario, Luigi, Bowser, Toad y Peach), donde cada corredor avanza a velocidades aleatorias y el orden de llegada es completamente impredecible en cada ejecución.

    \vspace{1cm}
    \begin{center}
        \textbf{\Huge Corredores} 
        \vspace{0.5cm}
        
        \begin{tabular}{@{}c@{}}\includegraphics[height=3.2cm]{images/mario.jpg}\\\textbf{Mario}\end{tabular}\hfill
        \begin{tabular}{@{}c@{}}\includegraphics[height=3.2cm]{images/luigi.jpg}\\\textbf{Luigi}\end{tabular}\hfill
        \begin{tabular}{@{}c@{}}\includegraphics[height=3.2cm]{images/bowser.jpg}\\\textbf{Bowser}\end{tabular}\hfill
        \begin{tabular}{@{}c@{}}\includegraphics[height=3.2cm]{images/toad.jpg}\\\textbf{Toad}\end{tabular}\hfill
        \begin{tabular}{@{}c@{}}\includegraphics[height=3.2cm]{images/peach.jpg}\\\textbf{Peach}\end{tabular}
    \end{center}

    \vspace{1cm}
    \begin{tcolorbox}[colback=green!3!white,colframe=blue!60!black,fonttitle=\bfseries]
        Cada corredor es una instancia de la clase \textbf{Coche}, que extiende \underline{Thread}. Al ejecutar el método \textbf{start()}, los cinco hilos corren en paralelo, compitiendo por llegar primero a la meta.
    \end{tcolorbox}

    \vspace{0.5cm}
    Gracias al uso de hilos y la aleatoriedad en los tiempos de sueño y velocidad, \textbf{cada carrera tiene un resultado diferente}, simulando perfectamente la emoción y la incertidumbre de una verdadera partida de Mario Kart.

% ========================== PLANTEAMIENTO ==========================
\newpage
\section{Planteamiento}
Inicialmente el proyecto era un programa de terminal que representaba mediante =, el avance de cada coche en la carrera.

\vspace{0.5cm}
Una vez realizado, se propuso el ejercicio de realizarlo mediante interfaz gráfica. Así que lo primero es plantear el programa:

\begin{itemize}
    \item Interfaz gráfica con \textbf{JavaFX}
    \item Cinco \textbf{Label} que muestran en tiempo real el recorrido de cada coche
    \item Un \textbf{ImageView} por corredor que cambia al llegar a la meta (medalla de oro, plata, bronce, cuarta posición y último)
    \begin{center} 
        \vspace{0.5cm}
        
        \begin{tabular}{@{}c@{}}\includegraphics[height=2cm]{images/oro.png}\\\textbf{Oro}\end{tabular}\hfill
        \begin{tabular}{@{}c@{}}\includegraphics[height=2cm]{images/plata.png}\\\textbf{Plata}\end{tabular}\hfill
        \begin{tabular}{@{}c@{}}\includegraphics[height=2cm]{images/bronce.png}\\\textbf{Bronce}\end{tabular}\hfill
        \begin{tabular}{@{}c@{}}\includegraphics[height=2cm]{images/hoja.png}\\\textbf{4º puesto}\end{tabular}\hfill
        \begin{tabular}{@{}c@{}}\includegraphics[height=2cm]{images/goomba.png}\\\textbf{5º puesto}\end{tabular}
    \end{center}
    \item Avance visual mediante cadenas de caracteres.
\end{itemize}

La diferencia principal respecto al programa de terminal, fue el tener que garantizar que las actualizaciones de la interfaz gráfica se realizaran desde hilos secundarios, mediante la utilización de \textbf{Platform.runLater()}.

% ========================== TECNOLOGÍAS UTILIZADAS ==========================
\newpage
\section{Tecnologías utilizadas}

\begin{itemize}
    \item \textbf{Java SE} – Lenguaje principal
    \item \textbf{JavaFX} – Interfaz gráfica y manejo de eventos
    \item \textbf{Hilos (Thread)} – Ejecución concurrente de los corredores
    \item \textbf{Maven/IntelliJ IDEA} – Gestión y desarrollo del proyecto
    \item \textbf{FXML} – Diseño declarativo de la interfaz
\end{itemize}

% ========================== DESCRIPCIÓN TÉCNICA ==========================
\newpage
\section{Uso del synchronized}
    \begin{tcolorbox}[colback=green!3!white,colframe=blue!60!black,fonttitle=\bfseries]
       
    \end{tcolorbox}

% ========================== PROBLEMAS SURGIDOS ==========================
\newpage
\section{Problemas surgidos y soluciones}

\begin{enumerate}
    \item \textbf{Actualización de la GUI desde hilos secundarios}
    
    \textit{Solución:} Uso obligatorio de \texttt{Platform.runLater()}
    
    \item \textbf{Sincronización del orden de llegada} 
    
    \textit{Solución:} Clase \texttt{Carrera} con contador atómico y lista ordenada
    
    \item \textbf{Botón clickable durante la carrera}
    
    \textit{Solución:} Deshabilitar el botón y usar \texttt{join()} en hilo auxiliar para reactivarlo al final
    
    \item \textbf{Exceso de salida en consola}
    
    \textit{Solución:} Comentada en versión final para mejor experiencia de usuario
\end{enumerate}

% ========================== CONCLUSIÓN ==========================
\newpage
\section{Conclusión}

Este proyecto ha permitido aplicar de forma práctica y divertida conceptos clave de la asignatura:

\begin{itemize}
    \item Creación y gestión de hilos en Java
    \item Concurrencia y comportamiento no determinista
    \item Comunicación segura entre hilos y la interfaz gráfica (JavaFX)
    \item Diseño de aplicaciones interactivas con eventos
\end{itemize}

El resultado es una carrera visual, dinámica y adictiva que refleja fielmente la esencia de Mario Kart, demostrando que con pocos elementos se puede crear una experiencia muy entretenida.

¡Es imposible no pulsar el botón de "REINICIAR" una y otra vez para ver quién gana esta vez!

% ========================== BIBLIOGRAFÍA ==========================
\newpage
\section{Bibliografía}

\begin{itemize}
    \item Oracle Docs – The Java™ Tutorials: Concurrency \\
    \url{https://docs.oracle.com/javase/tutorial/essential/concurrency/}
    \item OpenJFX Documentation \\
    \url{https://openjfx.io/}
    \item Stack Overflow – "JavaFX Platform.runLater and Task" (varias respuestas útiles)
    \item Imágenes de personajes y medallas extraídas del universo Mario Kart (Nintendo)
\end{itemize}

\end{document}