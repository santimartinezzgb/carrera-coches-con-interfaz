\documentclass[12pt]{article}
\usepackage[utf8]{inputenc}
\usepackage{geometry}
\geometry{margin=2cm}
\usepackage{graphicx}
\usepackage{hyperref}
\usepackage{media9}
\usepackage{listings}
\usepackage{xcolor}
\usepackage{tcolorbox}
\usepackage{url}



\begin{document}

    % ========================== PORTADA ==========================
    \begin{titlepage}
        \centering
        {\Large \textbf{Programación de Servicios y Procesos}\par}
        \vspace{2cm}
        {\Huge \textbf{Emulador de Mario Kart}\par}
        \vspace{0.5cm}
        {\LARGE Carrera de coches con Interfaz Gráfica en JavaFX\par}
        \vspace{2cm}
        \noindent\rule{12cm}{0.8pt}

        \vspace{2cm}
        {\Large \textbf{Autor:}\par}
        \vspace{0.3cm}
        {\large Santi Martínez\par}
        \vspace{10cm}
        {\large \today}
    \end{titlepage}

    \newpage
    \renewcommand{\contentsname}{ÍNDICE}
    \tableofcontents

    % ========================== INTRODUCCIÓN ==========================
    \newpage
    \section{Introducción}
        Este proyecto consiste en un emulador simplificado de una carrera de \textbf{Mario Kart} desarrollado en Java utilizando \textbf{JavaFX} para la interfaz gráfica y los conceptos fundamentales de \textbf{programación concurrente} con hilos (\texttt{Thread}).

        \vspace{0.5cm}
        El objetivo principal es simular una carrera entre cinco icónicos personajes del universo Mario Kart (Mario, Luigi, Bowser, Toad y Peach), donde cada corredor avanza a velocidades aleatorias y el orden de llegada es completamente impredecible en cada ejecución.

        \vspace{1cm}
        \begin{center}
            \textbf{\Huge Corredores} 
            \vspace{0.5cm}
            
            \begin{tabular}{@{}c@{}}\includegraphics[height=3.2cm]{images/mario.jpg}\\\textbf{Mario}\end{tabular}\hfill
            \begin{tabular}{@{}c@{}}\includegraphics[height=3.2cm]{images/luigi.jpg}\\\textbf{Luigi}\end{tabular}\hfill
            \begin{tabular}{@{}c@{}}\includegraphics[height=3.2cm]{images/bowser.jpg}\\\textbf{Bowser}\end{tabular}\hfill
            \begin{tabular}{@{}c@{}}\includegraphics[height=3.2cm]{images/toad.jpg}\\\textbf{Toad}\end{tabular}\hfill
            \begin{tabular}{@{}c@{}}\includegraphics[height=3.2cm]{images/peach.jpg}\\\textbf{Peach}\end{tabular}
        \end{center}

        \vspace{1cm}
        \begin{tcolorbox}[colback=green!3!white,colframe=blue!60!black,fonttitle=\bfseries]
            Cada corredor es una instancia de la clase \textbf{Coche}, que extiende \underline{Thread}. Al ejecutar el método \textbf{start()}, los cinco hilos corren en paralelo, compitiendo por llegar primero a la meta.
        \end{tcolorbox}

        \vspace{0.5cm}
        Gracias al uso de hilos y la aleatoriedad en los tiempos de sueño y velocidad, \textbf{cada carrera tiene un resultado diferente}, simulando perfectamente la emoción y la incertidumbre de una verdadera partida de Mario Kart.

    % ========================== PLANTEAMIENTO ==========================
    \newpage
    \section{Planteamiento}
        Inicialmente el proyecto era un programa de terminal que representaba mediante "=", el avance de cada coche en la carrera.

        \vspace{0.5cm}
        Una vez realizado, se propuso el ejercicio de realizarlo mediante interfaz gráfica. Así que lo primero es plantear el programa:

        \begin{itemize}
            \item Interfaz gráfica con \textbf{JavaFX}
            \item Cinco \textbf{Label} que muestran en tiempo real el recorrido de cada coche
        
            \item Avance visual mediante cadenas de caracteres.
        \end{itemize}

        \begin{figure}[!h]
            \begin{center}
                \includegraphics[height=6cm]{images/planteamiento.jpg}
            \end{center}
        \end{figure}

        La diferencia principal respecto al programa de terminal, fue el tener que garantizar que las actualizaciones de la interfaz gráfica se realizaran desde hilos secundarios, mediante la utilización de \textbf{Platform.runLater()}.

        \vspace{0.5cm}
        \begin{tcolorbox}[colback=green!3!white,colframe=blue!60!black,fonttitle=\bfseries]
            \textbf{Platform.runLater()} ejecuta código en el hilo de la interfaz gráfica de JavaFX. Su utilización se enfoca en actualizar la UI desde otro hilo para evitar errores de concurrencia.
        \end{tcolorbox}


    % ========================== DESCRIPCIÓN TÉCNICA ==========================
    \newpage
    \section{Uso del synchronized}
        \begin{tcolorbox}[colback=green!3!white,colframe=blue!60!black,fonttitle=\bfseries]
        \textbf{Synchronized} en Java garantiza que solo un hilo acceda a un bloque de código o método a la vez, evitando condiciones de carrera y problemas de concurrencia al sincronizar el acceso a recursos compartidos.
        \end{tcolorbox}

        \begin{figure}[!h]
            \begin{center}
                \includegraphics[height=9cm]{images/ejemplo-synchronized.png}
            \end{center}
        \end{figure}

        En este caso, el synchronized se utilizó para el método \textbf{ordenDeLlegada()}, el cual gestiona la llegada de cada instancia de la clase coche que iba finalizando el hilo, para así devolver la posición en carrera de los respectivos coches.

        \begin{figure}[!h]
            \begin{center}
                \includegraphics[height=7cm]{images/return-ordenDeLlegada().png}
            \end{center}
        \end{figure}

    % ========================== PROBLEMAS SURGIDOS ==========================
    \newpage
    \section{Problemas surgidos y soluciones}

    \begin{enumerate}
        \item \textbf{Actualización de la UI desde hilos secundarios}
        
        \textit{Solución:} Uso obligatorio de \texttt{Platform.runLater()}
        
        \item \textbf{Variable expected} 
        
        El problema que más me ha llevado tiempo del programa. Estoy escribiendo esto sin todavía haber llegado a una solución, así que se va a mostrar a continuación una captura del mismo:
        \begin{figure}[!h]
            \begin{center}
                \includegraphics[height=7cm]{images/variable-expected.png}
            \end{center}
        \end{figure}

        \textbf{Solución 1:} Se estaba intentando asignar un valor al resultado de un método, no a una variable; además de utilizar un getter y no un setter para añadir un valor. De ahí el primer fallo.

        \vspace{0.5cm}
        \textbf{Solución final:} No pude implementar lo que quería, que era un objeto Image a través de setUrl(), pero como ese método no existe en el objeto Image, debido a que no puede modificarse, cambié de rumbo en la idea que tenía en esta parte del programa.
        
        \newpage
        \item \textbf{Return de ordenDeLlegada en clase Carrera}: El problema fue que sin razón aparente, el método ordenDeLlegada(), necesario para que devuelva ese orden y trabajar con ese dato en la UI, se devolvía de dos en dos.
        \begin{figure}[!h]
            \begin{center}
                \includegraphics[height=7cm]{images/return-ordenDeLlegada().png}
            \end{center}
            \caption{Devuelve: 2, 4, 6, 8, 10}
        \end{figure}    
        
        \textit{Solución:} Problema con mútliples llamadas al método en la clase \textbf{Coche}.
        \begin{figure}[!h]
            \begin{center}
                \includegraphics[height=5cm]{images/solucion-ordenDeLlegada.png}
            \end{center}
        \end{figure}  
        
        Al realizar controles por terminal, se utilizó un System.out para comprobar qué número devolvía, sin darme cuenta que ese mismo control de errores causaba el error.

        Esto provocaba la suma doble del contador, debido a que ya se había llamado con anterioridad al método para modificar el texto de otro campo.

    \end{enumerate}

    % ========================== CONCLUSIÓN ==========================
    \newpage
    \section{Conclusión}

    El tema elegido para esta representación gráfica ha sido el \textbf{Mario Kart}. Es un juego que siempre me ha gustado y me pareció que podría adoptar una vista acorde para un programa de carrera de coches.

    \vspace{0.5cm}
    Pese a que el backend estaba ya prácticamente hecho del anterior trabajo, esta implementación me ha servido para poder estructurar mejor en mi cabeza el funcionamiento de los hilos y lo que pasa cuando comparten métodos con \underline{synchronized}.

    \vspace{0.5cm}
    La implementación gráfica al programa me ha parecido bastante entretenida. Creo que se puede deber a que estoy empezando a organizar todo desde el principio antes de empezar a teclear, y eso me está haciendo disfrutar mucho más de la programación.


    % ========================== BIBLIOGRAFÍA ==========================
    \newpage
    \section{Bibliografía}

    \begin{itemize}

        \item Imágenes de los personajes de Mario Kart \href{https://es.pinterest.com/}{Imágenes de personajes}
        \item https://stackoverflow.com/questions/10292792/getting-image-from-url-java
    \end{itemize}

\end{document}